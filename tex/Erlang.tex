\documentstyle{jarticle}
% \documentstyle[twocolumn]{jarticle}
\input epsf
\input comment.sty

\topmargin = 0cm
\oddsidemargin = 0cm
\evensidemargin = 0cm
\textheight = 24cm
\textwidth = 17cm

\setcounter{secnumdepth}{2}

%-----------------------------
\begin{document}

\title{\Large\bf  Erlang  }

\author{$H_2O$}

% \eauthor{
%  Katsumi Maruyama, Kazuya Hidaka, Soichiro Hidaka, Hirotatsu Hashizume, 
% }

% \affiliation{国立情報学研究所   National Institute of Informatics}

\maketitle

\section{出自}

\begin{itemize}

\item エリクソンの電子交換機プログラムの研究部門
\item 最初は Prolog応用から始まった
\item ATM交換機、DBMSなど実用システムに使われている。
     (関数型プログラムは魅力的だが、履歴依存システムを関数で記述するのは困難と思っていた。
      それを覆したのだから素晴らしい。)
\item 関数型言語 \\
      関数型言語には、Cf. ML, Haskel,,,, と色々あるけれど、実システム開発ではErlangが唯一。
\item 動的タイピング言語     \\
   Cf.  静的タイピング言語
\item 並列処理 \\
      メッセージを交換しあうプロセスの集まり。 非同期通信。
\item  CSPとも通ずるモデル
\item  マルチコアに容易に対応
\item 分散処理
\end{itemize}

  
\section{関数型言語}

\begin{itemize}
    \item  副作用のない関数が基本
    \item  宣言型
    \item  単一代入
    \item  Tail recursion
        Cf.   \verb| void foo(int i) { .....  foo(i+1);} | を実行したらスタックはどうなるか?
    
    \item  状態の表現
    \item  関数は副作用を持たない
   \item  状態依存の処理:  Tail recursionで表現。
\end{itemize}


\section{データタイプ}

\subsection{プリミティブデータタイプ}

       atom

       Integer

       Float

       Pid   プロセスの識別子

       Ref

       fun   関数  
    
\subsection{複合データタイプ}


\begin{description}
    \item  [タプル  tuples]    要素数は不変。要素のタイプは実行時に決まる。\\
        \verb| {a, 12, 34} |
    \item  [リスト   list]     様相数は可変。要素のタイプは実行時に決まる。 \\
        \verb|    [a, 23, hi, lo] |
    \item  [レコード(構造体)]   フィールド名を付けたタプル
        \begin{verbatim}
            -record(Name, {FieldName1, ,,,, FieldNameN}.  
            #Name.FieldName   
        \end{verbatim}

\end{description}


\subsection{変数}

      〇変数はCapital letterで始まる\\

\subsection{パターンマッチ}

なかなか使いやすい。

\begin{verbatim}
       パターン = 式 
       Ex.  {A, B} = {12, apple}
            {C, [Head | Tail]} = {{222, man},[a, b, c]}
\end{verbatim}



\section{シーケンシャル処理}

\begin{verbatim}
    〇  Factorial
      -module(math).
      -export([fac/1]).
      fac(N) when N > 0 -> N * fac(N-1);
      fac(0)            -> 1.
      
      > math:fac(25).
    
    〇 binary tree lookup 
      lookup(Key, {Key, Val, _, _})  ->  {ok, Val};
      lookup(Key, {Key1, Val, S, B}) when Key < Key1  ->  lookup(Key, S);
      lookup(Key, {Key1, Val, S, B})  ->  lookup(Key, B);
      lookup(Key, nil)  ->  not_found.
    
    〇リストのアペンド
      append([H | T], L)  ->  [H | append(T, L)];
      append([], L)       ->  L.
    
    〇 List member 
      member(H, [H | _])  ->  true;
      member(H, [_ | T])  ->  member(H, T);
      member(_, [])       ->  false.

    〇 Case 式   
      case Expr  of
           Pattern1 [when ガード1]  -> Seq1;
           Pattern2 [when ガード2]  -> Seq2;
        ...
       end;
    
    〇 If 式
       if 
          Guard1 -> Sequence1;
       Guard2 -> Sequence2;
          ...
       end
\end{verbatim}    


\section{ 関数オブジェクト}

\begin{verbatim}
      fun(Arg1,.., ArgN)  ->  ....... end.
    
    〇 関数も First class 値なので、パラメータや返値として記述できる。
    〇 定義例
      K = 2,
      F = fun(X) -> X * K end.        %% 変数Fの値は関数

    〇 定義例
      adder(C)  ->  fun(X) -> X + C end.    %% addr(C)は関数を返す
      > Add10 = adder(10).
      #Fun
      > Add10(8).
      18
    
    〇 Lisp のマップ機能
      map(F, [H | T])   ->  [F(H) | map(T)];
      map(F, [])        ->  [].
      
      > map(Add10, [1,2,3,4,5]).
      [11,12,13,14,15]
    
    〇リスト内包  (comprehension)
      [Term || P1, P2,,, Pn]
           where Pi is Pattern <- Exporesion or
                       prediate.
    
    〇 quick ソート
      sort([X | Xs])  ->
        sort([Y || Y <- Xs, Y < X])  ++
        [X] ++
        sort([Y || Y <- Xs, Y >= X]);
        sort([])       ->  [].
        
      ++: append operator
\end{verbatim}

    
\section{コンカレント処理}

{\bf (1) 基本}

\begin{verbatim}
    〇プロセスの生成
         spawn(echo, loop, [])
    
    〇サーバの例
      -module(echo).
      -export([start/0, loop/0]).
      
      start()  ->  spawn(echo, loop, []).
    
        loop() ->
           receive
             {From, Message} ->
                      From ! Message,
                      loop()
           end.
    
      ...
      Pid = echo:start(),
      Pid ! {self(), hello}
      ....
    
    〇メッセージ受信: パターンマッチングが行われる
         receive
         Message1 -> 
         ......;
         Message2 ->
          ......
        end
\end{verbatim}



{\bf (2) 銀行システムの例}

\begin{verbatim}
      -module(bank_server).
      -export([start/0, server/1]).
      
      start()  -> register(bank_server, spawn(bank_server, server, [[]])).
      
      server(Data)  ->
              receive
                 {From, {ask, Who}}  ->
                     reply(From, lookup(Who, Data)),
                     server(Data);         %% Tail recursion。状態を引き継ぐ。
                 {From, {deposit, Who, Amount}} -> 
                     reply(From, ok),
                     server(deposit(Who, Amount, Data));
                 {From, {Withdraw, Who, Amount}} ->
                     case  lookup(Who, Data) of 
                          undefined ->
                                  reply(From, no),
                                  server(Data);   %% Tail recursion。状態を引き継ぐ。
                           Balance  when Balance > Amount  ->
                                  reply(From, ok),
                                  server(deposit(Who, -Amount, Data)); %% Tail recursion。
                           _   ->
                                  reply(From, no),
                                  server(Data)   %% Tail recursion。状態を引き継ぐ。
                     end
                 end.
      
      reply(To, X)  -> To ! {bank_server, X}.
      
      lookup(Who, [{Who, Value} | _])   ->  Value;
      lookup(Who, [_ | T])              ->  lookup(Who, T);
      lookup(_, _)                      ->  undefined.
      
      deposit(Who, X, [{Who, Balance} | T]) -> [{Who, Balance + X} | T];
      deposit(Who, X [H | T]) ->   [H | deposit(Who, X, T)];
      deposit(Who, X, [])  ->  [{Who, X}].
      
      
      -------------------------------------
      -module(bank_client).
      -export([ask/1, deposit/2, withdraw]).
      
      ask(Who)  ->  rpc({ask, Who}).
      deposit(Who, Amount)  ->  rpc({deposit, Who, Amount}).
      withdraw(Who, Amount) -> rpc({withdraw, Who, Amount}).
      
      rpc(Msg)  ->
          bank_server ! {self(), Msg},
          receive
               {bank_server, Reply} -> Reply
          end.

\end{verbatim} 
    

\section{分散処理}

\begin{verbatim}
    〇リモートノード上にプロセスを生成し、通信できる。
      Pid = spawn(Node, Module, Func, ArgList).
    
        bank_server ! {self(), Msg},
    
        {bank_server, 'host@domainname'} ! { self(), Msg}
\end{verbatim}



\section{mapreduce}

  Google の map-reduceアルゴリズムも簡潔に記述できる。\\
  マルチコア向き。

\begin{verbatim}
【例】
    -module(mapredudemdl).
    -export([mapreduce/4]).
    
    mapreduce(F1, F2, Acc0, l) ->
        S = self(),
        Pid = spawn(fun() -> reducew(S, F1, F2, Acc0, L) end),
        receive
            (Pid, Result) -> Result
        end.
    
    reduce(Parent, F1, F2, Acc0, L) ->
        process_flag(trap_exit, true),
        ReducePid = self(),
        foreach(fun(X) -> spawn_link(fun() -> do_job(ReducePid, F1, X) end) end, L),
        N = length(L),
        Dict0 = dict:new(),
        Dict1 = collect_replies(N, Dict0),
        Acc = dict:fold(F2, Acc0, Dict1),
        Parent ! { self(), Acc }.
    
    collect_replies(0, Dist) -> Dict;
    collect_replies(N, Dict) ->
        receive
            {Key, Val} -> case dict:is_key(Key, Dict) of
                              true -> Dict1 = dict:append(Key, Val, Dict),
                                      collect_replies(N, Dict1);
                              false-> Dict1 = dist:store(Key, [Val], Dict),
                                      collect_replies(N, Dict1)
                          end;
           {'EXIT', _, Why} -> collect_replies(N-1, Dict)
        end.
    
    do_job(ReducePid, F, X) -> F(ReducePid, X).
    
\end{verbatim}


\section{こんなことをやりたかった}

。。。。。。。。。。。。。


\end{document}
